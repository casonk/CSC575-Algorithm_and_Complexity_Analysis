\documentclass[12pt]{article}

\usepackage{amsthm}
\usepackage{amsmath}
\usepackage{amssymb}
\usepackage{mathtools}
\usepackage{xcolor}
\usepackage{graphicx}
\usepackage{pgfplots}
\usepackage{hyperref}
\usepackage{url}

\usepackage[left = 1cm, top = 2cm, bottom = 3cm, right = 1cm]{geometry}

\newcommand{\XB}{\color{black}}
\newcommand{\XBB}{\color{blue}}
\newcommand{\XV}{\color{violet}}
\newcommand{\XR}{\color{red}}
\newcommand{\ds}{\displaystyle}

\begin{document}

\title{\textbf{CSC575}: Algorithm \& Complexity Analysis - Reflection}
\date{\today}
\author{\XV\textit{\large{\href{https://github.com/casonk}{Cason Konzer}}}\XB}

\maketitle
\hrulefill
\vfill 
    % \underline{Key Concepts}: .

\newpage

\section{Summative Reflection}
\begin{itemize}
    % \item \textbf{In what ways have you improved as a scientist? What brought about those improvements? Point to specific experiences, readings, or discussions in this course.}
    % \item \textbf{What did you learn by creating the Workshop? Be specific with respect to your work on the assignment and the topics/skills/concepts you learned in the course.}
    \item \textbf{What was your biggest accomplishment in the course? How did the Workshop and other course elements help you reach it? Be specific.}
    % \item \textbf{What skills have you mastered in this course? How are they reflected in the Workshop? Be specific.}
\end{itemize}

From my perspective the biggest accomplishment I have in this course is understanding the complexity class NP, NPC, and algorithms which are NP hard. A first introduction of these ideas were given by the instructor in class which gave a solid background. My knowledge of the topic was further extended by completing the assigned readings in addition to sections from the sentinel work `Computers and Intractability' by Michael R. Garey and David S. Johnson. Rounding off this accomplishment was workshop 3 in which I read and watched additional works on reductions of various problems and studied in depth the reduction from the \(k-clique\) problem to the maximality of \(\gamma-quasi-cliques\). Furthering this is the presentation of such reduction from $3-CNF-SAT$ to $k-clique$ which forced me to develop a deeper understanding of the techniques and identify my prior ignorance. 

\section{Process Reflection}
\begin{itemize}
    % \item \textbf{What problems did you encounter in preparing the Workshops? How did you troubleshoot them, if you did?}
    % \item \textbf{Talk about the aims and strategies that led to the completion of your Workshops. How did your thinking about it evolve over time (point to specific experiences while working on the assignment)? How did the assignment evolve (or not evolve) with your thinking (again, point to specific experiences) about it? What went according to plan and what surprises did you encounter? What still needs work?}
    % \item \textbf{What risks did you take in the assignment/course? Be specific.}
    \item \textbf{Outline the steps you took to complete the Workshops and Peer Reviews and tell me about your thinking at each step.}
    % \item \textbf{Write about your learning process throughout this course and what it felt like at different stages until you mastered certain skills. Discuss skills you are still developing.}
\end{itemize}

When working thorough the workshops and peer reviews my very first step was to read the instructions, twice, and ask any questions on which I needed clarification. After I had a clear understanding of the task at hand I usually started with more or less a brainstorming task. For the algorithm development workshops (1 and 2) this consisted of drawing out the problem in one note, while in workshop 3 this consisted of identifying relevant problems I face currently in my work and research which were candidates and selecting a candidate based on knowledge gain priority and available literature. Once I had a clear understanding of the (candidate) problem I would create a rough draft framework and present on this 3-5 times to refine it to an acceptable and comprehensive framework (for algorithm workshops this additionally concerns improvements and simplifications of the algorithm). Once I was happy with the framework I would record the presentation, to the best of my ability, as a one take while having the framework available on a separate screen, and post-edit as needed. I took time to upload supporting material in multiple locations for the students who wised to take a deeper dive into my work. At this point the final take was uploaded and I proceeded to peer reviews which consisted of watching the student presentations in an iterative manner. I watched each students lecture 3-5 time in which each time I took notes and examined exactly what it was that they were presenting, reading supplementary materials as provided. I gave the students then detailed feedback based on the provided rubric, identifying pitfalls anf fallacies within their work while additionally providing suggestions for improvement. 

\section{Evaluative Reflection}
\begin{itemize}
    \item \textbf{What are the strengths and weaknesses of your Workshops and Peer Reviews? What did you hope for students to learn from your work? What more did you want to accomplish? How could you approach a future Workshop or Peer Review differently?}
\end{itemize}

First for my workshops I will identify the strengths of which material is presented in a whiteboard lecture style manner and is self contained. This mode makes the presentation understandable for a broad range of audiences with various backgrounds (I think even the undergraduate students would have been able to follow the works). As mentioned in the process reflection such a mode was enabled by an iterative process. This mode of presentation also creates weaknesses in the workshops for audiences who prefer a slide deck or otherwise regard the self contained material as trivial. Additionally, such a one take styled presentation presented some tangential thoughts as you would typically experience in lecture of conversation. At times within the workshops, given I had pre-pasted text from the frameworks, it could be small to read and/or blurry to the viewer, I would attribute this to both that the video was often slightly cropped and that youtube's auto quality if often not the maximum thus reducing resolution. My hope for all of the workshops was that students would have a clear understanding of both the computational/mathematical and general concepts of the presentation such that they could both implement the algorithms and explain to others the proofs given. I wished for workshops 1 and 2 additional improvements upon the algorithms time and space complexity while in workshop 3 additional time to present on the algorithm (separate from the proof) in which the paper presented. For future approaches I would make the presentations less self contained and make the assumption that the students have adequate background to understand the more trivial aspects. Additionally I would be sure to limit text width to ensure it is clearly visible to the audience (a problem less relevant for a physical board). 

\section{Reflection on Learning}
\begin{itemize}
    % \item \textbf{Make connections between what you studied in this Theory of Computation course with what you’ve learned in other courses at UM-Flint or before. Make specific references to your work in this class and in the other courses. How did what you learn in the other courses enhance what you learned in here, and vice versa?}
    \item \textbf{What does your work on the Workshop illustrate about you as a learner? Be specific.}
    % \item \textbf{Reflect on how you thought about Theory of Computation before you took this course and how you think about it now. Have any of your assumptions or understandings changed? Why?}
    % \item \textbf{Outline a chronology of significant events and “Aha” moments during the semester. Why do these events and “Aha” moments stand out? How did they help you develop as a student inside and outside of the class?}
\end{itemize}

Building upon the previous reflections I can say that the workshops highlight some key elements of myself as a learner. The first of these topics is that for both the workshops and peer reviews I take time within the assignments to reflect on my self evaluation via an iterative approach. What I can mention is that while clearly my first iterations are never sufficient, even my final iterations have room for additional improvement (of course everyone in practice has a quality and effort threshold which guides their final iteration). Building upon this is that I strive to obtain a comprehensive understanding of both my own work and the works presented by others. If there is something not required of me which I am confident on my understanding of I try to take the time to provide the audience with the additional information. In a similar manner, if there is material which I am not confident on I will omit it to reduce confusion to the audience. Lastly, I have personally identified many pitfalls with powerpoint styled presentations in which I have noticed in the past, such as a higher tendency to gloss over important topics and a reduced opportunity for knowledge transfer of these materials. It is for this reason that I find myself, in the aspect as a teacher, adequate by providing a more hands on and tangible experience which is clear and more easily grasped by the audience. 

\section{Interpretive Reflection}
\begin{itemize}
    \item \textbf{Write a very short personal narrative (story) about the time you spent creating the workshops and conducting the peer reviews for this course. Include musing, scene setting, and dialogue. Include both your perspective as you created the assignment and your perspective once the assignment was submitted.  Consider the progression from one workshop to the next.}
    % \item \textbf{Given your experience with algorithm analysis and design, what are the best parts about it?  What are the negative aspects which you believe cannot be changed, and why are they foundational to the process?  What are the negative aspects which you believe could be changed, and how would you change them? }
\end{itemize}

In the most honest manner I describe my time spent in workshop creation and peer reviews. Initial discussion of upcoming assignments start in the class sessions, eager to see the problem definition and stressed on time due to a busy schedule, I look forward to the assignment posting to get an early start. At the end of the night (soon thereafter) I get a notification of the assignment posting and take a first look at the definition on my phone. The following day, after work, I read again the assigned problem and pay a closer attention to the requirements. At this time I do some initial brainstorming and take notes on the topic. Iteratively I start to build a framework, day after day, until I feel it is sufficient for an initial recording. I have created a presentation which I am happy with, but as I review again the next day I am overwhelmed by the errors I have noticed. Again I revise my framework, clarify any misunderstandings, and re-record the presentation with a focus on improving upon the prior pitfalls. This process I continue not until it is perfect, but until I have ran out of the drive for continued improvement due to the piling tasks in work and research which I have been putting on the back-burner. At this point I take the time to give a final review and post-edit the presentation, often late into the night. I upload the presentation to youtube, finalize all repositories of the supplementary material, and post my submission to the class portal. 

Once the reviews are posted I give each presentation an initial watch and write draft reviews. Later on I read again the reviews, watch again the presentations, and revise them. Often within the review process I realize the first times I have watched there was areas in which I have misinterpreted, or simply did not understand, and thus find myself often pausing slides and working thorough the explicit material with a virtual pen. At this point I baseline against my own work, often I realize more areas in which I could have improved from the criticisms and areas of improvement I have mentioned. Again I watch the video and revise the reviews to be more realistic against the baseline standards outlined in the requirements and what I feel is expected of the students. At the end of the reviews I have learned something new, a different take on the problem at hand, or a different problem altogether. As always there is more to do, but this life is limited and time is precious, I wish we had more. 

Across the workshops I found myself holding the expectation of the presentation quality to a similar standard. With this said, contrary to the expectation, I find myself less dedicated to the tasks as the semester goes on. Granted, I find the material increasingly complex and my understanding deepening. I attribute this not to the class format but rather to the parallel tasks I have at hand. Throughout the semester my full time job has been ramping up with the new year and the tasks have been increasing. I am extremely relieved the semester is coming to and end and excited to have some more time in which I can dedicate to the life part of work-life balance.  

\end{document}